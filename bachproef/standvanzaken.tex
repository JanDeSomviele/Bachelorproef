\chapter{\IfLanguageName{dutch}{Stand van zaken}{State of the art}}%
\label{ch:stand-van-zaken}

% Tip: Begin elk hoofdstuk met een paragraaf inleiding die beschrijft hoe
% dit hoofdstuk past binnen het geheel van de bachelorproef. Geef in het
% bijzonder aan wat de link is met het vorige en volgende hoofdstuk.

% Pas na deze inleidende paragraaf komt de eerste sectiehoofding.

Dit hoofdstuk bevat je literatuurstudie. De inhoud gaat verder op de inleiding, maar zal het onderwerp van de bachelorproef *diepgaand* uitspitten. De bedoeling is dat de lezer na lezing van dit hoofdstuk helemaal op de hoogte is van de huidige stand van zaken (state-of-the-art) in het onderzoeksdomein. Iemand die niet vertrouwd is met het onderwerp, weet nu voldoende om de rest van het verhaal te kunnen volgen, zonder dat die er nog andere informatie moet over opzoeken \autocite{Pollefliet2011}.

Je verwijst bij elke bewering die je doet, vakterm die je introduceert, enz.\ naar je bronnen. In \LaTeX{} kan dat met het commando \texttt{$\backslash${textcite\{\}}} of \texttt{$\backslash${autocite\{\}}}. Als argument van het commando geef je de ``sleutel'' van een ``record'' in een bibliografische databank in het Bib\LaTeX{}-formaat (een tekstbestand). Als je expliciet naar de auteur verwijst in de zin (narratieve referentie), gebruik je \texttt{$\backslash${}textcite\{\}}. Soms is de auteursnaam niet expliciet een onderdeel van de zin, dan gebruik je \texttt{$\backslash${}autocite\{\}} (referentie tussen haakjes). Dit gebruik je bv.~bij een citaat, of om in het bijschrift van een overgenomen afbeelding, broncode, tabel, enz. te verwijzen naar de bron. In de volgende paragraaf een voorbeeld van elk.

\textcite{Knuth1998} schreef een van de standaardwerken over sorteer- en zoekalgoritmen. Experten zijn het erover eens dat cloud computing een interessante opportuniteit vormen, zowel voor gebruikers als voor dienstverleners op vlak van informatietechnologie~\autocite{Creeger2009}.

Let er ook op: het \texttt{cite}-commando voor de punt, dus binnen de zin. Je verwijst meteen naar een bron in de eerste zin die erop gebaseerd is, dus niet pas op het einde van een paragraaf.


Bij het schrijven van een literatuurstudie over portals en platforms, kun je de volgende grote hoofdstukken opnemen en onderverdelen in relevante subsecties om een goed gestructureerd en gedetailleerd onderzoek te presenteren:

\section 1. Wat is een Portaal/Platform?
#### 1.1 Definitie van een Portaal en Platform
- **Portaal**: Een overzicht van wat een portaal inhoudt, inclusief functies zoals informatieaggregatie, gebruikersauthenticatie en single sign-on.
- **Platform**: Uitleg van wat een platform is, inclusief de technische en zakelijke aspecten.

#### 1.2 Typen Portalen en Platforms
- **Consumentenportalen**: Zoals nieuwsportalen, e-commerce websites.
- **Bedrijfsportalen**: Zoals intranetportalen, klantportalen.
- **Technologische Platforms**: Zoals cloud computing platforms, ontwikkelplatforms (bijv. AWS, Google Cloud).

#### 1.3 Voorbeelden en Casestudies
- Concrete voorbeelden van populaire portalen en platforms, hun gebruik en impact.

\section 2. Waarom Gebruik van een Platform Beter is?
#### 2.1 Voordelen van Platforms
- **Efficiëntie en Schaalbaarheid**: Hoe platforms operationele efficiëntie en schaalbaarheid bevorderen.
- **Ecosysteem en Netwerkeffecten**: De voordelen van het bouwen en deelnemen aan een platformecosysteem.
- **Innovatie en Samenwerking**: Hoe platforms innovatie en samenwerking tussen verschillende partijen stimuleren.

#### 2.2 Economische en Technische Voordelen
- **Kostenreductie**: Besparingen op infrastructuur, onderhoud en ontwikkeling.
- **Snellere Time-to-Market**: Versnelling van productontwikkeling en -lancering.

#### 2.3 Gebruiksgemak en Gebruikerservaring
- **Integratie**: Gemakkelijke integratie met bestaande systemen.
- **Gebruikersvriendelijkheid**: Verbeterde gebruikerservaring door gestandaardiseerde interfaces.

\section 3. Open Source Platforms
#### 3.1 Definitie en Voordelen van Open Source
- **Transparantie en Veiligheid**: Open broncode voor inspectie en verbetering.
- **Community Support**: Voordelen van een actieve gemeenschap die bijdraagt aan de ontwikkeling en onderhoud.

#### 3.2 Voorbeelden van Open Source Platforms
- **Spotify**: Overzicht van de technologie en de platformbenadering van Spotify, inclusief hun gebruik van open source technologieën.
- **Crowdsourcing Platforms**: Hoe crowdsourcing platforms werken en hun impact, met voorbeelden zoals Wikipedia en OpenStreetMap.

#### 3.3 Open Source in Bedrijfsomgevingen
- **Adoptie en Implementatie**: Voorbeelden van bedrijven die open source gebruiken en hoe ze het hebben geïmplementeerd.
- **Succesverhalen en Uitdagingen**: Case studies van succesvolle open source projecten en de uitdagingen die ermee gepaard gaan.

\section 4. Backstage.io als Platform
#### 4.1 Introductie van Backstage.io
- **Wat is Backstage.io?**: Definitie en doelen van het platform.
- **Geschiedenis en Ontwikkeling**: Hoe en waarom Backstage.io is ontwikkeld door Spotify.

 4.2 Functionaliteiten en Voordelen van Backstage.io
- **Architectuur**: De technische architectuur van Backstage.io.
- **Functionaliteiten**: Overzicht van de belangrijkste functionaliteiten, zoals service catalogus, software templates, en plug-in ecosysteem.
- **Voordelen**: Hoe Backstage.io bedrijven helpt met het organiseren en beheren van hun technologische ecosystemen.

 4.3 Implementatie en Gebruikscases
- **Implementatie**: Stappen en beste praktijken voor het implementeren van Backstage.io in een organisatie.
- **Succesverhalen**: Voorbeelden van organisaties die Backstage.io met succes hebben geïmplementeerd en de voordelen die ze hebben ervaren.

Deze hoofdstukken en subsecties vormen samen een uitgebreide literatuurstudie die diep ingaat op de verschillende aspecten van portalen en platforms, hun voordelen, het belang van open source oplossingen, en een specifieke case study van Backstage.io. Zorg ervoor dat elk hoofdstuk goed onderbouwd is met relevante literatuur, voorbeelden en case studies om de argumenten en bevindingen te ondersteunen.


\textcite{MEANS2011} Een platform in deze context wordt beschreven als herbruikbare middelen verbonden aan een product of technologie, inclusief het ontwikkelings- en productieproces. Volgens Robertsen en Ulrich draait een productplatform om het spreiden van ontwikkelingskosten over meerdere producten die dezelfde subsystemen delen. Voor bedrijven zonder geschikte productfamilies om componenten te delen, zoals leveranciers van maatwerk, ligt de nadruk op kennisdeling op een hoger abstractieniveau. Technologieplatforms zijn ontwikkeld om uitdagingen aan te gaan in diverse productportfolio’s waar componenten niet kunnen worden hergebruikt.


\textcite{manysoftwarefirmwareproductstoday1996} In de tekst wordt het concept van een platform ontwikkeld door HP beschreven als een samenhangende architectuur die veel verder gaat dan een eenvoudige herbruikbibliotheek. Platforms bestaan uit verschillende subsystemen of componenten die als basis dienen voor productontwikkeling en zijn ontworpen om de ontwikkeltijd van producten aanzienlijk te verkorten door hergebruik van bestaande structuren en functionaliteiten. Het gebruik van een platform biedt aanzienlijke voordelen, zoals een kortere time-to-market (TTM) voor producten, doordat het platform zorgt voor hergebruik van basisstructuren en -componenten. Dit leidt tot een efficiënter ontwikkelingsproces. Platforms bieden ook de mogelijkheid om nieuwe functionaliteiten eerst op productniveau te testen en later, na stabilisatie en marktacceptatie, naar het platform te migreren, wat de ontwikkelcyclus optimaliseert. De beschrijving van HP’s platformontwikkeling geeft inzicht in de noodzaak van een duidelijke en expliciete platformarchitectuur. Dit is vergelijkbaar met hoe Backstage.io werkt, waarbij een gecentraliseerd en gedocumenteerd systeem de samenwerking en integratie binnen ontwikkelteams verbetert. Het expliciet vastleggen van leveringsmodellen en het gebruik van hybride push-pull mechanismen voor componentleveringen binnen HP illustreert de noodzaak van goed gedefinieerde processen en structuren die ook voor Backstage.io van toepassing zijn.

\textcite{Wang_2015} Een portaal of platform, zoals TRUSTIE, is een geïntegreerd systeem ontworpen om softwareontwikkeling te faciliteren en te verbeteren door gebruik te maken van crowdsourcing. Het biedt tools en diensten om de samenwerking tussen interne ontwikkelteams en externe ontwikkelaars te bevorderen, waardoor de kwaliteit en productiviteit van softwareprojecten wordt verhoogd. TRUSTIE ondersteunt verschillende aspecten van softwareontwikkeling, zoals eisenbeheer, ontwerp, pakketverwerking en onderhoud, en combineert deze met crowdsourcing-principes om een bredere deelname en innovatie mogelijk te maken.

\textcite{Perks_2017} Het gebruik van een platform biedt aanzienlijke voordelen omdat het de waardecreatie binnen netwerken stimuleert en optimaliseert. Platforms helpen leidende bedrijven om toekomstige netwerkwaarde te identificeren en te communiceren, waardoor partners worden aangetrokken en gemotiveerd. Ze bevorderen innovativiteit door middelen en kennis vrij te delen, wat netwerkleden aanzet tot het ontdekken en ontwikkelen van nieuwe waarde. Platforms bouwen ook legitimiteit op binnen en buiten het netwerk door duidelijke prestatie-indicatoren te communiceren, waardoor vertrouwen en investeringen toenemen. Bovendien betrekken ze het netwerk bij organisatorische aanpassingen, wat flexibiliteit en aanpassingsvermogen vergroot. Door deze geïntegreerde benadering realiseren platforms meer dynamische en effectieve samenwerking en waardecreatie.

\textcite{Riehle_2016} Het toepassen van open-sourceprincipes binnen bedrijven, bekend als inner source, biedt diverse voordelen voor platformgebaseerde productontwikkeling. Door samenwerking en kennisdeling tussen verschillende teams en afdelingen te bevorderen, vermindert inner source organisatorische silo’s. Dit leidt tot snellere innovatie doordat ontwikkelaars zelf problemen kunnen oplossen en nieuwe functies kunnen toevoegen zonder op goedkeuring van andere teams te wachten. Bovendien verhoogt inner source de ontwikkelings-efficiëntie door hergebruik van bestaande componenten en het verbeteren van codekwaliteit. Producteenheden kunnen actiever deelnemen aan de prioritering van vereisten, wat resulteert in betere besluitvorming en een lager werklast voor platformorganisaties. Tot slot verbetert inner source de werktevredenheid van ontwikkelaars door hen de mogelijkheid te geven hun vaardigheden en bijdragen zichtbaar te maken binnen het bedrijf.

\textcite{Reza_Gharehbagh_2022} Multi-sided platforms (MSPs) significantly enhance green technology development (GTD) for capital-constrained manufacturing entrepreneurs. They provide access to financing and retailing facilities, crucial for entrepreneurs pursuing environmental innovation under government policies. MSPs with balanced power structures lead to improved GTD outcomes, fostering win-win agreements among stakeholders. This highlights the pivotal role of platforms in facilitating sustainable development and environmental innovation, making them essential for achieving GTD goals efficiently.

Uit \textcite{Muffatto_2000}  worden volgende bevindingen geïllustreerd: de samenvatting geeft inzichten uit een literatuurstudie over verschillende benaderingen van productontwikkeling, met een focus op het gebruik van platforms als organisatiestructuur. Case A, genaamd Carmake, toont hoe een platformteam als hoofdorganisatiestructuur wordt gebruikt voor productontwikkeling, waarbij co-locatie van teamleden wordt benadrukt als een middel om teamwork en samenwerking te bevorderen. Case B, Scrapers Co., daarentegen heeft geen platformteams, maar vertrouwt op gemeenschappelijke productplanning en supervisie voor standaardisatie. In Case C, Whitegoods, wordt ook een platformbenadering gebruikt, maar met een meer gedecentraliseerde organisatiestructuur. Een synoptische tabel illustreert de verschillen in teammanagement en organisatie tussen de cases, met nadruk op permanente versus tijdelijke teams, co-locatie en componententeams. Het belang van het beheren en behouden van kennis binnen teams wordt benadrukt, vooral bij gebruik van permanente teams, terwijl ook wordt gewezen op de uitdagingen van middelenmobiliteit binnen projectstructuren. De bevindingen uit deze literatuurstudie benadrukken de verschillende benaderingen die organisaties kunnen gebruiken bij het implementeren van platformgestuurde productontwikkeling, waarbij de organisatiestructuur een cruciale rol speelt in het succes van dergelijke initiatieven.



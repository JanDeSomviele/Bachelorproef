\chapter{Plugins}
\label{ch:plugins}

\section{Backstage plugins}

Backstage is een open-source platform voor het bouwen van ontwikkelaarsplatformen, gemaakt door Spotify. Het is ontworpen om softwareontwikkeling en infrastructuurbeheer te stroomlijnen door een enkele interface te bieden voor alle tools, diensten en documentatie. Centraal in de flexibiliteit en kracht van Backstage staat de plugin-architectuur. Plugins in Backstage zorgen voor de uitbreiding van de functionaliteit van het platform, waardoor teams verschillende tools en diensten naadloos kunnen integreren.

\subsection{Backend/Frontend plugins}

Backend plugins van Backstage communiceren met externe services, databases en API's en voeren taken uit zoals het ophalen, verwerken en beheren van gegevens.Backend plugins worden meestal geschreven in Node.js en maken gebruik van dezelfde technologie stack als Backstage backend.

Frontend plugins zijn verantwoordelijk voor de gebruikersinterface en interactie binnen de Backstage webapplicatie. Ze worden gebouwd met React en integreren met de hoofd-UI van Backstage om een samenhangende gebruikerservaring te bieden. Met deze plugins kunnen gebruikers gegevens visualiseren, communiceren met backend services en verschillende taken direct vanuit het Backstage-portaal uitvoeren.

\subsection{Het maken van Backstage plugins}

Het maken van een Backstage plugin bestaat uit verschillende stappen. Het eerste is het opzetten van een plugin door de Backstage CLI te gebruiken om deze te genereren. Dit zorgt voor de opzet van de basis bestandsstructuur en boilerplate code. Vervolgens definieer je de plugincomponenten door de kernfunctionaliteit te implementeren. Dit houdt in dat je React componenten maakt voor de frontend en Node.js modules voor de backend. Integreer vervolgens de plugin met Backstage door deze te registreren met de applicatie, routes, navigatie en eventueel benodigde backend services te configureren. De laatste stap is het testen van de plugin en het schrijven van de documentatie voor gebruikers.

\subsection{Structuur van een plugin}

Een typische Backstage plugin heeft een gestructureerde organisatie. De `src/` map bevat frontend componenten, hooks en API interacties. De `backend/` map bevat backend logica, inclusief service definities en API routes. Het `plugin.ts` bestand dient als het belangrijkste ingangspunt voor de plugin, want het registreert de plugin met Backstage. Het `package.json` bestand bevat afhankelijkheden en scripts voor het bouwen en uitvoeren van de plugin. De `README.md` bevat documentatie en gebruiksinstructies.

\subsection{Composability System}

Het composability system in Backstage zorgt ervoor dat plugins modulair en herbruikbaar zijn. Dit systeem is gebaseerd op het concept van samenstelbare componenten die kunnen worden samengevoegd tot complete applicaties. De belangrijkste elementen zijn componentcompositie, waarbij plugins componenten blootleggen die kunnen worden hergebruikt en gecombineerd met andere componenten, wat herbruikbaarheid en consistentie bevordert. Uitbreidingspunten zijn vooraf gedefinieerde punten in de applicatie waarop plugins kunnen inhaken en functionaliteit kunnen uitbreiden. API's en services worden ook aangeboden en gebruikt door plugins, waardoor interactie mogelijk is tussen verschillende plugins en services.

\subsection{Internationalization}

Internationalisatie (i18n) in Backstage is essentieel voor wereldwijde bedrijven die in meerdere talen en regio's werken. Door meerdere talen te ondersteunen, kan de gebruikersinterface worden aangepast aan de voorkeurstaal van elke gebruiker, wat de bruikbaarheid en toegankelijkheid verbetert. Hierdoor kunnen teams in verschillende regio's hetzelfde platform gebruiken zonder taalbarrières, wat de samenwerking bevordert. 

Het implementeren van i18n in Backstage omvat het maken van lokalisatiebestanden die vertalingen voor verschillende talen bevatten, het gebruik van hooks die door Backstage worden geleverd om de juiste vertalingen op te halen en weer te geven op basis van gebruikersvoorkeuren, en het configureren van Backstage voor het beheren van en schakelen tussen verschillende lokalisaties.

\subsection{Plugin analytics}

Plugin analytics in Backstage bieden inzicht in hoe plugins worden gebruikt en helpen teams om datagestuurde beslissingen te nemen. Gebruiksmeting houdt bij hoe vaak plugins worden gebruikt, welke functies het populairst zijn en hoeveel gebruikers erbij betrokken zijn. Prestatiemonitoring meet de prestaties van plugins om knelpunten of problemen te identificeren en aan te pakken. Foutopsporing logt en analyseert fouten om de stabiliteit en betrouwbaarheid van plugins te verbeteren.

Het integreren van analytics houdt in dat plugins worden geïnstrueerd door instructie code toe te voegen om specifieke gebeurtenissen en statistieken te volgen. Deze gegevens worden verzameld en naar een analytics service gestuurd voor opslag en analyse. De inzichten worden gevisualiseerd door dashboards en rapporten.

\subsection{Feature Flags}

Feature flags in Backstage maken gecontroleerde rollouts en experimenten mogelijk door features dynamisch in of uit te schakelen. Dit biedt verschillende voordelen. Met geleidelijke uitrol kunnen nieuwe functies geleidelijk worden geïntroduceerd bij een subset van gebruikers om de impact te controleren en problemen vroegtijdig op te sporen. A/B-tests kunnen worden uitgevoerd om verschillende implementaties te vergelijken en de beste te kiezen op basis van feedback van gebruikers en prestaties. Met directe rollbacks kunnen functies die problemen veroorzaken snel worden uitgeschakeld zonder nieuwe code te implementeren.

Het implementeren van feature flags omvat het definiëren van flags voor nieuwe of experimentele features. Het gebruik van voorwaardelijke logica in de code om de status van feature flags te controleren en het gedrag overeenkomstig aan te passen. Het beheren van flags via een feature flag management service om de status van flags te controleren en het gebruik ervan te monitoren.

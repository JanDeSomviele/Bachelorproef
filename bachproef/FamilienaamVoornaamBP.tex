%===============================================================================
% LaTeX sjabloon voor de bachelorproef toegepaste informatica aan HOGENT
% Meer info op https://github.com/HoGentTIN/latex-hogent-report
%===============================================================================

\documentclass[dutch,dit,thesis]{hogentreport}

% TODO:
% - If necessary, replace the option `dit`' with your own department!
%   Valid entries are dbo, dbt, dgz, dit, dlo, dog, dsa, soa
% - If you write your thesis in English (remark: only possible after getting
%   explicit approval!), remove the option "dutch," or replace with "english".

\usepackage{lipsum} % For blind text, can be removed after adding actual content

%% Pictures to include in the text can be put in the graphics/ folder
\graphicspath{{graphics/}}

%% For source code highlighting, requires pygments to be installed
%% Compile with the -shell-escape flag!
\usepackage[section]{minted}
%% If you compile with the make_thesis.{bat,sh} script, use the following
%% import instead:
%% \usepackage[section,outputdir=../output]{minted}
\usemintedstyle{solarized-light}
\definecolor{bg}{RGB}{253,246,227} %% Set the background color of the codeframe

%% Change this line to edit the line numbering style:
\renewcommand{\theFancyVerbLine}{\ttfamily\scriptsize\arabic{FancyVerbLine}}

%% Macro definition to load external java source files with \javacode{filename}:
\newmintedfile[javacode]{java}{
    bgcolor=bg,
    fontfamily=tt,
    linenos=true,
    numberblanklines=true,
    numbersep=5pt,
    gobble=0,
    framesep=2mm,
    funcnamehighlighting=true,
    tabsize=4,
    obeytabs=false,
    breaklines=true,
    mathescape=false
    samepage=false,
    showspaces=false,
    showtabs =false,
    texcl=false,
}

% Other packages not already included can be imported here

%%---------- Document metadata -------------------------------------------------
% TODO: Replace this with your own information
\author{Ernst Aarden}
\supervisor{Dhr. F. Van Houte}
\cosupervisor{Mevr. S. Beeckman}
\title[Optionele ondertitel]%
    {Titel van de bachelorproef}
\academicyear{\advance\year by -1 \the\year--\advance\year by 1 \the\year}
\examperiod{1}
\degreesought{\IfLanguageName{dutch}{Professionele bachelor in de toegepaste informatica}{Bachelor of applied computer science}}
\partialthesis{false} %% To display 'in partial fulfilment'
%\institution{Internshipcompany BVBA.}

%% Add global exceptions to the hyphenation here
\hyphenation{back-slash}

%% The bibliography (style and settings are  found in hogentthesis.cls)
\addbibresource{bachproef.bib}            %% Bibliography file
\addbibresource{../voorstel/voorstel.bib} %% Bibliography research proposal
\defbibheading{bibempty}{}

%% Prevent empty pages for right-handed chapter starts in twoside mode
\renewcommand{\cleardoublepage}{\clearpage}

\renewcommand{\arraystretch}{1.2}

%% Content starts here.
\begin{document}

%---------- Front matter -------------------------------------------------------

\frontmatter

\hypersetup{pageanchor=false} %% Disable page numbering references
%% Render a Dutch outer title page if the main language is English
\IfLanguageName{english}{%
    %% If necessary, information can be changed here
    \degreesought{Professionele Bachelor toegepaste informatica}%
    \begin{otherlanguage}{dutch}%
       \maketitle%
    \end{otherlanguage}%
}{}

%% Generates title page content
\maketitle
\hypersetup{pageanchor=true}

%%=============================================================================
%% Voorwoord
%%=============================================================================

\chapter*{\IfLanguageName{dutch}{Woord vooraf}{Preface}}%
\label{ch:voorwoord}

%% TODO:
%% Het voorwoord is het enige deel van de bachelorproef waar je vanuit je
%% eigen standpunt (``ik-vorm'') mag schrijven. Je kan hier bv. motiveren
%% waarom jij het onderwerp wil bespreken.
%% Vergeet ook niet te bedanken wie je geholpen/gesteund/... heeft


%%=============================================================================
%% Samenvatting
%%=============================================================================

% TODO: De "abstract" of samenvatting is een kernachtige (~ 1 blz. voor een
% thesis) synthese van het document.
%
% Een goede abstract biedt een kernachtig antwoord op volgende vragen:
%
% 1. Waarover gaat de bachelorproef?
% 2. Waarom heb je er over geschreven?
% 3. Hoe heb je het onderzoek uitgevoerd?
% 4. Wat waren de resultaten? Wat blijkt uit je onderzoek?
% 5. Wat betekenen je resultaten? Wat is de relevantie voor het werkveld?
%
% Daarom bestaat een abstract uit volgende componenten:
%
% - inleiding + kaderen thema
% - probleemstelling
% - (centrale) onderzoeksvraag
% - onderzoeksdoelstelling
% - methodologie
% - resultaten (beperk tot de belangrijkste, relevant voor de onderzoeksvraag)
% - conclusies, aanbevelingen, beperkingen
%
% LET OP! Een samenvatting is GEEN voorwoord!

%%---------- Nederlandse samenvatting -----------------------------------------
%
% TODO: Als je je bachelorproef in het Engels schrijft, moet je eerst een
% Nederlandse samenvatting invoegen. Haal daarvoor onderstaande code uit
% commentaar.
% Wie zijn bachelorproef in het Nederlands schrijft, kan dit negeren, de inhoud
% wordt niet in het document ingevoegd.

\IfLanguageName{english}{%
\selectlanguage{dutch}
\chapter*{Samenvatting}
\lipsum[1-4]
\selectlanguage{english}
}{}

%%---------- Samenvatting -----------------------------------------------------
% De samenvatting in de hoofdtaal van het document

\chapter*{\IfLanguageName{dutch}{Samenvatting}{Abstract}}

\lipsum[1-4]


%---------- Inhoud, lijst figuren, ... -----------------------------------------

\tableofcontents

% In a list of figures, the complete caption will be included. To prevent this,
% ALWAYS add a short description in the caption!
%
%  \caption[short description]{elaborate description}
%
% If you do, only the short description will be used in the list of figures

\listoffigures

% If you included tables and/or source code listings, uncomment the appropriate
% lines.
%\listoftables
%\listoflistings

% Als je een lijst van afkortingen of termen wil toevoegen, dan hoort die
% hier thuis. Gebruik bijvoorbeeld de ``glossaries'' package.
% https://www.overleaf.com/learn/latex/Glossaries

%---------- Kern ---------------------------------------------------------------

\mainmatter{}

% De eerste hoofdstukken van een bachelorproef zijn meestal een inleiding op
% het onderwerp, literatuurstudie en verantwoording methodologie.
% Aarzel niet om een meer beschrijvende titel aan deze hoofdstukken te geven of
% om bijvoorbeeld de inleiding en/of stand van zaken over meerdere hoofdstukken
% te verspreiden!

%%=============================================================================
%% Inleiding
%%=============================================================================

\chapter{\IfLanguageName{dutch}{Inleiding}{Introduction}}%
\label{ch:inleiding}

De inleiding moet de lezer net genoeg informatie verschaffen om het onderwerp te begrijpen en in te zien waarom de onderzoeksvraag de moeite waard is om te onderzoeken. In de inleiding ga je literatuurverwijzingen beperken, zodat de tekst vlot leesbaar blijft. Je kan de inleiding verder onderverdelen in secties als dit de tekst verduidelijkt. Zaken die aan bod kunnen komen in de inleiding~\autocite{Pollefliet2011}:

\begin{itemize}
  \item context, achtergrond
  \item afbakenen van het onderwerp
  \item verantwoording van het onderwerp, methodologie
  \item probleemstelling
  \item onderzoeksdoelstelling
  \item onderzoeksvraag
  \item \ldots
\end{itemize}

\section{\IfLanguageName{dutch}{Probleemstelling}{Problem Statement}}%
\label{sec:probleemstelling}

Uit je probleemstelling moet duidelijk zijn dat je onderzoek een meerwaarde heeft voor een concrete doelgroep. De doelgroep moet goed gedefinieerd en afgelijnd zijn. Doelgroepen als ``bedrijven,'' ``KMO's'', systeembeheerders, enz.~zijn nog te vaag. Als je een lijstje kan maken van de personen/organisaties die een meerwaarde zullen vinden in deze bachelorproef (dit is eigenlijk je steekproefkader), dan is dat een indicatie dat de doelgroep goed gedefinieerd is. Dit kan een enkel bedrijf zijn of zelfs één persoon (je co-promotor/opdrachtgever).

\section{\IfLanguageName{dutch}{Onderzoeksvraag}{Research question}}%
\label{sec:onderzoeksvraag}

Wees zo concreet mogelijk bij het formuleren van je onderzoeksvraag. Een onderzoeksvraag is trouwens iets waar nog niemand op dit moment een antwoord heeft (voor zover je kan nagaan). Het opzoeken van bestaande informatie (bv. ``welke tools bestaan er voor deze toepassing?'') is dus geen onderzoeksvraag. Je kan de onderzoeksvraag verder specifiëren in deelvragen. Bv.~als je onderzoek gaat over performantiemetingen, dan 

\section{\IfLanguageName{dutch}{Onderzoeksdoelstelling}{Research objective}}%
\label{sec:onderzoeksdoelstelling}

Wat is het beoogde resultaat van je bachelorproef? Wat zijn de criteria voor succes? Beschrijf die zo concreet mogelijk. Gaat het bv.\ om een proof-of-concept, een prototype, een verslag met aanbevelingen, een vergelijkende studie, enz.

\section{\IfLanguageName{dutch}{Opzet van deze bachelorproef}{Structure of this bachelor thesis}}%
\label{sec:opzet-bachelorproef}

% Het is gebruikelijk aan het einde van de inleiding een overzicht te
% geven van de opbouw van de rest van de tekst. Deze sectie bevat al een aanzet
% die je kan aanvullen/aanpassen in functie van je eigen tekst.

De rest van deze bachelorproef is als volgt opgebouwd:

In Hoofdstuk~\ref{ch:stand-van-zaken} wordt een overzicht gegeven van de stand van zaken binnen het onderzoeksdomein, op basis van een literatuurstudie.

In Hoofdstuk~\ref{ch:methodologie} wordt de methodologie toegelicht en worden de gebruikte onderzoekstechnieken besproken om een antwoord te kunnen formuleren op de onderzoeksvragen.

% TODO: Vul hier aan voor je eigen hoofstukken, één of twee zinnen per hoofdstuk

In Hoofdstuk~\ref{ch:conclusie}, tenslotte, wordt de conclusie gegeven en een antwoord geformuleerd op de onderzoeksvragen. Daarbij wordt ook een aanzet gegeven voor toekomstig onderzoek binnen dit domein.
\chapter{\IfLanguageName{dutch}{Stand van zaken}{State of the art}}%
\label{ch:stand-van-zaken}

% Tip: Begin elk hoofdstuk met een paragraaf inleiding die beschrijft hoe
% dit hoofdstuk past binnen het geheel van de bachelorproef. Geef in het
% bijzonder aan wat de link is met het vorige en volgende hoofdstuk.

% Pas na deze inleidende paragraaf komt de eerste sectiehoofding.

%Dit hoofdstuk bevat je literatuurstudie. De inhoud gaat verder op de inleiding, maar zal het onderwerp van de bachelorproef *diepgaand* uitspitten. De bedoeling is dat de lezer na lezing van dit hoofdstuk helemaal op de hoogte is van de huidige stand van zaken (state-of-the-art) in het onderzoeksdomein. Iemand die niet vertrouwd is met het onderwerp, weet nu voldoende om de rest van het verhaal te kunnen volgen, zonder dat die er nog andere informatie moet over opzoeken \autocite{Pollefliet2011}.
%
%Je verwijst bij elke bewering die je doet, vakterm die je introduceert, enz.\ naar je bronnen. In \LaTeX{} kan dat met het commando \texttt{$\backslash${textcite\{\}}} of \texttt{$\backslash${autocite\{\}}}. Als argument van het commando geef je de ``sleutel'' van een ``record'' in een bibliografische databank in het Bib\LaTeX{}-formaat (een tekstbestand). Als je expliciet naar de auteur verwijst in de zin (narratieve referentie), gebruik je \texttt{$\backslash${}textcite\{\}}. Soms is de auteursnaam niet expliciet een onderdeel van de zin, dan gebruik je \texttt{$\backslash${}autocite\{\}} (referentie tussen haakjes). Dit gebruik je bv.~bij een citaat, of om in het bijschrift van een overgenomen afbeelding, broncode, tabel, enz. te verwijzen naar de bron. In de volgende paragraaf een voorbeeld van elk.
%
%\textcite{Knuth1998} schreef een van de standaardwerken over sorteer- en zoekalgoritmen. Experten zijn het erover eens dat cloud computing een interessante opportuniteit vormen, zowel voor gebruikers als voor dienstverleners op vlak van informatietechnologie~\autocite{Creeger2009}.
%
%Let er ook op: het \texttt{cite}-commando voor de punt, dus binnen de zin. Je verwijst meteen naar een bron in de eerste zin die erop gebaseerd is, dus niet pas op het einde van een paragraaf.
%

\section{Wat is een portaal/platform?}

\section{Voordelen van een portaal/platform}

\section{Open Source en crowdsourcing}



%\textcite{MEANS2011} Een platform in deze context wordt beschreven als herbruikbare middelen verbonden aan een product of technologie, inclusief het ontwikkelings- en productieproces. Volgens Robertsen en Ulrich draait een productplatform om het spreiden van ontwikkelingskosten over meerdere producten die dezelfde subsystemen delen. Voor bedrijven zonder geschikte productfamilies om componenten te delen, zoals leveranciers van maatwerk, ligt de nadruk op kennisdeling op een hoger abstractieniveau. Technologieplatforms zijn ontwikkeld om uitdagingen aan te gaan in diverse productportfolio’s waar componenten niet kunnen worden hergebruikt.


%\textcite{manysoftwarefirmwareproductstoday1996} In de tekst wordt het concept van een platform ontwikkeld door HP beschreven als een samenhangende architectuur die veel verder gaat dan een eenvoudige herbruikbibliotheek. Platforms bestaan uit verschillende subsystemen of componenten die als basis dienen voor productontwikkeling en zijn ontworpen om de ontwikkeltijd van producten aanzienlijk te verkorten door hergebruik van bestaande structuren en functionaliteiten. Het gebruik van een platform biedt aanzienlijke voordelen, zoals een kortere time-to-market (TTM) voor producten, doordat het platform zorgt voor hergebruik van basisstructuren en -componenten. Dit leidt tot een efficiënter ontwikkelingsproces. Platforms bieden ook de mogelijkheid om nieuwe functionaliteiten eerst op productniveau te testen en later, na stabilisatie en marktacceptatie, naar het platform te migreren, wat de ontwikkelcyclus optimaliseert. De beschrijving van HP’s platformontwikkeling geeft inzicht in de noodzaak van een duidelijke en expliciete platformarchitectuur. Dit is vergelijkbaar met hoe Backstage.io werkt, waarbij een gecentraliseerd en gedocumenteerd systeem de samenwerking en integratie binnen ontwikkelteams verbetert. Het expliciet vastleggen van leveringsmodellen en het gebruik van hybride push-pull mechanismen voor componentleveringen binnen HP illustreert de noodzaak van goed gedefinieerde processen en structuren die ook voor Backstage.io van toepassing zijn.

%\textcite{Wang_2015} Een portaal of platform, zoals TRUSTIE, is een geïntegreerd systeem ontworpen om softwareontwikkeling te faciliteren en te verbeteren door gebruik te maken van crowdsourcing. Het biedt tools en diensten om de samenwerking tussen interne ontwikkelteams en externe ontwikkelaars te bevorderen, waardoor de kwaliteit en productiviteit van softwareprojecten wordt verhoogd. TRUSTIE ondersteunt verschillende aspecten van softwareontwikkeling, zoals eisenbeheer, ontwerp, pakketverwerking en onderhoud, en combineert deze met crowdsourcing-principes om een bredere deelname en innovatie mogelijk te maken.

%\textcite{Perks_2017} Het gebruik van een platform biedt aanzienlijke voordelen omdat het de waardecreatie binnen netwerken stimuleert en optimaliseert. Platforms helpen leidende bedrijven om toekomstige netwerkwaarde te identificeren en te communiceren, waardoor partners worden aangetrokken en gemotiveerd. Ze bevorderen innovativiteit door middelen en kennis vrij te delen, wat netwerkleden aanzet tot het ontdekken en ontwikkelen van nieuwe waarde. Platforms bouwen ook legitimiteit op binnen en buiten het netwerk door duidelijke prestatie-indicatoren te communiceren, waardoor vertrouwen en investeringen toenemen. Bovendien betrekken ze het netwerk bij organisatorische aanpassingen, wat flexibiliteit en aanpassingsvermogen vergroot. Door deze geïntegreerde benadering realiseren platforms meer dynamische en effectieve samenwerking en waardecreatie.

%\textcite{Riehle_2016} Het toepassen van open-sourceprincipes binnen bedrijven, bekend als inner source, biedt diverse voordelen voor platformgebaseerde productontwikkeling. Door samenwerking en kennisdeling tussen verschillende teams en afdelingen te bevorderen, vermindert inner source organisatorische silo’s. Dit leidt tot snellere innovatie doordat ontwikkelaars zelf problemen kunnen oplossen en nieuwe functies kunnen toevoegen zonder op goedkeuring van andere teams te wachten. Bovendien verhoogt inner source de ontwikkelings-efficiëntie door hergebruik van bestaande componenten en het verbeteren van codekwaliteit. Producteenheden kunnen actiever deelnemen aan de prioritering van vereisten, wat resulteert in betere besluitvorming en een lager werklast voor platformorganisaties. Tot slot verbetert inner source de werktevredenheid van ontwikkelaars door hen de mogelijkheid te geven hun vaardigheden en bijdragen zichtbaar te maken binnen het bedrijf.
%
%\textcite{Reza_Gharehbagh_2022} Multi-sided platforms (MSPs) significantly enhance green technology development (GTD) for capital-constrained manufacturing entrepreneurs. They provide access to financing and retailing facilities, crucial for entrepreneurs pursuing environmental innovation under government policies. MSPs with balanced power structures lead to improved GTD outcomes, fostering win-win agreements among stakeholders. This highlights the pivotal role of platforms in facilitating sustainable development and environmental innovation, making them essential for achieving GTD goals efficiently.
%
%Uit \textcite{Muffatto_2000}  worden volgende bevindingen geïllustreerd: de samenvatting geeft inzichten uit een literatuurstudie over verschillende benaderingen van productontwikkeling, met een focus op het gebruik van platforms als organisatiestructuur. Case A, genaamd Carmake, toont hoe een platformteam als hoofdorganisatiestructuur wordt gebruikt voor productontwikkeling, waarbij co-locatie van teamleden wordt benadrukt als een middel om teamwork en samenwerking te bevorderen. Case B, Scrapers Co., daarentegen heeft geen platformteams, maar vertrouwt op gemeenschappelijke productplanning en supervisie voor standaardisatie. In Case C, Whitegoods, wordt ook een platformbenadering gebruikt, maar met een meer gedecentraliseerde organisatiestructuur. Een synoptische tabel illustreert de verschillen in teammanagement en organisatie tussen de cases, met nadruk op permanente versus tijdelijke teams, co-locatie en componententeams. Het belang van het beheren en behouden van kennis binnen teams wordt benadrukt, vooral bij gebruik van permanente teams, terwijl ook wordt gewezen op de uitdagingen van middelenmobiliteit binnen projectstructuren. De bevindingen uit deze literatuurstudie benadrukken de verschillende benaderingen die organisaties kunnen gebruiken bij het implementeren van platformgestuurde productontwikkeling, waarbij de organisatiestructuur een cruciale rol speelt in het succes van dergelijke initiatieven.



%%=============================================================================
%% Methodologie
%%=============================================================================

\chapter{\IfLanguageName{dutch}{Methodologie}{Methodology}}%
\label{ch:methodologie}

%% TODO: In dit hoofstuk geef je een korte toelichting over hoe je te werk bent
%% gegaan. Verdeel je onderzoek in grote fasen, en licht in elke fase toe wat
%% de doelstelling was, welke deliverables daar uit gekomen zijn, en welke
%% onderzoeksmethoden je daarbij toegepast hebt. Verantwoord waarom je
%% op deze manier te werk gegaan bent.
%% 
%% Voorbeelden van zulke fasen zijn: literatuurstudie, opstellen van een
%% requirements-analyse, opstellen long-list (bij vergelijkende studie),
%% selectie van geschikte tools (bij vergelijkende studie, "short-list"),
%% opzetten testopstelling/PoC, uitvoeren testen en verzamelen
%% van resultaten, analyse van resultaten, ...
%%
%% !!!!! LET OP !!!!!
%%
%% Het is uitdrukkelijk NIET de bedoeling dat je het grootste deel van de corpus
%% van je bachelorproef in dit hoofstuk verwerkt! Dit hoofdstuk is eerder een
%% kort overzicht van je plan van aanpak.
%%
%% Maak voor elke fase (behalve het literatuuronderzoek) een NIEUW HOOFDSTUK aan
%% en geef het een gepaste titel.

\section{Literatuurstudie}

\section{Selectie plugins}

\section{Proof of Concept}

\section{Analyse van de Proof of Concept}


% Voeg hier je eigen hoofdstukken toe die de ``corpus'' van je bachelorproef
% vormen. De structuur en titels hangen af van je eigen onderzoek. Je kan bv.
% elke fase in je onderzoek in een apart hoofdstuk bespreken.

%\input{...}
%\input{...}
%...

%%=============================================================================
%% Conclusie
%%=============================================================================

\chapter{Conclusie}%
\label{ch:conclusie}

% TODO: Trek een duidelijke conclusie, in de vorm van een antwoord op de
% onderzoeksvra(a)g(en). Wat was jouw bijdrage aan het onderzoeksdomein en
% hoe biedt dit meerwaarde aan het vakgebied/doelgroep? 
% Reflecteer kritisch over het resultaat. In Engelse teksten wordt deze sectie
% ``Discussion'' genoemd. Had je deze uitkomst verwacht? Zijn er zaken die nog
% niet duidelijk zijn?
% Heeft het onderzoek geleid tot nieuwe vragen die uitnodigen tot verder 
%onderzoek?





%---------- Bijlagen -----------------------------------------------------------

\appendix

\chapter{Onderzoeksvoorstel}

Het onderwerp van deze bachelorproef is gebaseerd op een onderzoeksvoorstel dat vooraf werd beoordeeld door de promotor. Dat voorstel is opgenomen in deze bijlage.

%% TODO: 
%\section*{Samenvatting}

% Kopieer en plak hier de samenvatting (abstract) van je onderzoeksvoorstel.

% Verwijzing naar het bestand met de inhoud van het onderzoeksvoorstel
%---------- Inleiding ---------------------------------------------------------

\section{Introductie}%
\label{sec:introductie}
% TO DO: uitleg Backstage.io

De vraag naar dit onderzoek komt omdat Wolters Kluwer wilt overschakelen naar een cloud omgeving en hun software ontwikkelingsplatform eventueel wil aanpassen. Deze bachelorproef richt zich op de ontwikkeling van een intern ontwikkelingsplatform met behulp van Backstage.io. Het platform fungeert als een centrale hub voor diverse ontwikkelingsprocessen en -tools. De focus ligt op het optimaliseren van de ontwikkeling, het stroomlijnen van processen en het verbeteren van de algehele efficiëntie van interne softwareontwikkeling. 
De kernvraag van dit onderzoek is hoe Backstage.io kan worden geïmplementeerd en aangepast binnen Wolters Kluwer als intern ontwikkelingsplatform om de betrouwbaarheid te verbeteren en de soepele implementatie van softwaretoepassingen te faciliteren. Bovendien zal een proof of concept worden geleverd om de praktische toepassing en haalbaarheid van Backstage.io binnen de context van Wolters Kluwer weer te geven.




%Formuleer duidelijk de onderzoeksvraag! De begeleiders lezen nog steeds te veel voorstellen waarin we geen onderzoeksvraag terugvinden.


%---------- Stand van zaken ---------------------------------------------------

\section{State-of-the-art}%
\label{sec:state-of-the-art}
% TO DO:huidige toestand in WoKlu, in referenties documentatie en ev. docs in wolters kluwer van huidige toestand

\cite{CK_BackstageGithub}
\cite{Backstage_Docs}

In Wolters kluwer wordt er nu gebruikt gemaakt van TeamCity, Jira en Confluence voor het development process. Omdat het bedrijf wilt overschakelen naar een cloud omgeving in plaats van de on premises omgeving van teamcity zoeken ze alternatieven. Backstage.io is een van deze alternatieven. Backstage.io is een open source spotify programma dat gemakkelijke toegang biedt aan de resources van developers. 

%---------- Methodologie ------------------------------------------------------
\section{Methodologie}%
\label{sec:methodologie}
\subsection{Literatuurstudie}
In deze fase wordt een uitgebreide literatuurstudie uitgevoerd volgens de CRAAP-methode om bestaande implementatiepraktijken van Backstage.io als intern ontwikkelingsplatform te onderzoeken. Hierbij wordt specifiek gekeken naar succesvolle aanpassingen en integraties in vergelijkbare organisaties. Deze fase heeft als resultaat een literatuurstudierapport waarin relevante informatie over de implementatie van Backstage.io wordt behandeld.

\subsection{Oplossingsontwerp }
Voer gesprekken met ontwikkelaars en belanghebbenden om de behoeften en vereisten voor de implementatie van Backstage.io als intern ontwikkelingsplatform vast te stellen. Definieer duidelijk de functionaliteiten en aanpasbaarheidsmogelijkheden, waarbij ontwikkelaars de flexibiliteit hebben om het platform aan te passen aan specifieke implementatievereisten. Deze fase heeft als deliverable een gedetailleerd ontwerp van de implementatie.

\subsection{Implementatie:}
Implementeer Backstage.io met aandacht voor de ontworpen functionaliteiten en aanpasbaarheid. Integreer het platform in de bestaande ontwikkelingsprocessen van Wolters Kluwer. Deze fase heeft als deliverable een functioneel intern ontwikkelingsplatform met Backstage.io, klaar voor verdere evaluatie.

\subsection{Proof of Concept}
Pas het geïmplementeerde Backstage.io-platform toe in een real-world scenario om de functionaliteit, efficiëntie en aanpasbaarheid te valideren binnen de specifieke context van Wolters Kluwer. Verzamel feedback van ontwikkelaars en belanghebbenden over de bruikbaarheid en effectiviteit van het platform. Het proof of concept-rapport geeft concrete inzichten en verbeteringsmogelijkheden.

\subsection{Evaluatie}
Schrijf een gedetailleerd rapport met een beschrijving van het geïmplementeerde Backstage.io als intern ontwikkelingsplatform. In dit rapport wordt ook de feedback van de proof of concept geanalyseerd en worden aanbevelingen gedaan voor verdere optimalisatie en implementatie in de praktijk. Deze fase heeft als deliverable een eindrapport en een implementatievoorstel.

%---------- Verwachte resultaten ----------------------------------------------
\section{Verwacht resultaat, conclusie}%
\label{sec:verwachte_resultaten}

Deze bachelorproef streeft naar de ontwikkeling van een intern ontwikkelingsplatform met behulp van Backstage.io. Verwacht wordt dat dit platform aanzienlijke voordelen zal bieden, waaronder efficiëntere ontwikkelingsprocessen, aanpasbaarheid aan de unieke omgeving van de organisatie, verbeterde algehele efficiëntie van interne softwareontwikkeling, en een gestroomlijnde integratie van diverse ontwikkelingstools.

Het ontwikkelde platform zal concrete resultaten opleveren, waarbij de functionaliteit en voordelen van Backstage.io worden benadrukt in het eindrapport en het implementatievoorstel. Het is belangrijk op te merken dat de verwachtingen als leidraad dienen en dat het onderzoek openstaat voor nieuwe inzichten, met specifieke aandacht voor het begrijpen van ontwikkelingsprocessen binnen de context van het gebruik van Backstage.io.



%%---------- Andere bijlagen --------------------------------------------------
% TODO: Voeg hier eventuele andere bijlagen toe. Bv. als je deze BP voor de
% tweede keer indient, een overzicht van de verbeteringen t.o.v. het origineel.
%\input{...}

%%---------- Backmatter, referentielijst ---------------------------------------

\backmatter{}

\setlength\bibitemsep{2pt} %% Add Some space between the bibliograpy entries
\printbibliography[heading=bibintoc]

\end{document}

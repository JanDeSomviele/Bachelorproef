\chapter{Backstage}
\label{ch:Backstage}

Backstage.io is een open-source platform ontwikkeld door Spotify, ontworpen om de productiviteit van ontwikkelaars te verhogen en het beheer van interne infrastructuur te vereenvoudigen. Het biedt een gecentraliseerde interface voor het beheren van softwarecomponenten en tools, waardoor ontwikkelteams efficiënter kunnen werken. Dit hoofdstuk zal een diepgaande blik werpen op verschillende aspecten van Backstage.io, inclusief de softwarecatalogus, softwaretemplates, TechDocs en open-source plugins.

\section{Backstage Software Catalog}

De Backstage Software Catalog is een essentieel onderdeel van het platform en biedt een uitgebreide en gestandaardiseerde manier om alle softwarecomponenten binnen een organisatie te beheren en te visualiseren. Deze catalogus fungeert als een centrale opslagplaats voor al je softwarediensten, bibliotheken, gegevenspipelines en meer.

\subsection{Kenmerken en Functionaliteiten}

De softwarecatalogus centraliseert alle informatie over de softwarecomponenten binnen een organisatie, wat een duidelijk overzicht biedt van wat er beschikbaar is, wie verantwoordelijk is voor welk onderdeel, en hoe de verschillende componenten met elkaar verbonden zijn. Backstage stelt gebruikers in staat om uitgebreide metadata over elke softwarecomponent te beheren, inclusief informatie zoals eigendom, documentatie, afhankelijkheden en de huidige status van de component. De catalogus is doorzoekbaar, waardoor ontwikkelaars snel de benodigde componenten kunnen vinden, wat tijd bespaart en de efficiëntie van het ontwikkelproces verhoogt. Bovendien integreert Backstage met verschillende andere tools en diensten binnen je organisatie, van CI/CD-pipelines tot monitoringtools, waardoor een naadloze workflow ontstaat. De catalogus biedt ook mogelijkheden voor toegangscontrole, zodat alleen geautoriseerde gebruikers bepaalde informatie kunnen zien of wijzigen, wat cruciaal is voor het handhaven van de beveiliging en integriteit van je software-ecosysteem.

\subsection{Voordelen}

De softwarecatalogus van Backstage.io biedt aanzienlijke voordelen door transparantie en traceerbaarheid binnen een organisatie te verhogen. Door alle softwarecomponenten op één plek te beheren, kunnen teams beter samenwerken dankzij gedeelde informatie en gemeenschappelijke standaarden. Bovendien zorgt het gecentraliseerde overzicht ervoor dat ontwikkelaars snel relevante componenten en documentatie kunnen vinden, wat cruciaal is voor een snelle incidentrespons en probleemoplossing.

\section{Backstage Software Templates}

Backstage Software Templates zijn ontworpen om de creatie van nieuwe projecten en componenten te standaardiseren en te versnellen. Ze bieden een herbruikbare blauwdruk voor verschillende soorten softwarecomponenten, zodat teams snel aan de slag kunnen zonder telkens vanaf nul te hoeven beginnen.

\subsection{Kenmerken en Functionaliteiten}

Templates zorgen ervoor dat nieuwe projecten voldoen aan de vastgestelde normen en best practices van de organisatie, wat consistentie en kwaliteit bevordert. Hoewel templates gestandaardiseerd zijn, kunnen ze worden aangepast aan de specifieke behoeften van verschillende teams of projecten, wat flexibiliteit biedt zonder de basisprincipes van standaardisatie te compromitteren. Het gebruik van templates kan veel repetitieve taken automatiseren, zoals het opzetten van repositorystructuren, configuratiebestanden en basisimplementaties, wat tijd bespaart en de kans op fouten vermindert. Templates kunnen ook ingebouwde documentatie bevatten, zodat nieuwe projecten direct beschikken over de benodigde informatie om goed van start te gaan.

\subsection{Voordelen}

De softwaretemplates van Backstage.io dragen bij aan een snellere onboarding van nieuwe teamleden en projecten doordat zij gebruik kunnen maken van kant-en-klare templates. Door standaarden en best practices te implementeren in templates, zorg je voor een consistentere kwaliteit van de software. Bovendien verhoogt de automatisering van repetitieve taken de efficiëntie, waardoor ontwikkelaars zich kunnen concentreren op de kernfunctionaliteit van hun projecten.

\section{Backstage TechDocs}

TechDocs in Backstage biedt een manier om technische documentatie te beheren en te publiceren binnen dezelfde interface die wordt gebruikt voor het beheer van softwarecomponenten. Het is gebaseerd op Markdown-bestanden die in de repository van de betreffende componenten zijn opgeslagen.

\subsection{Kenmerken en Functionaliteiten}

TechDocs maakt gebruik van Markdown, een eenvoudige opmaaktaal die gemakkelijk te schrijven en te onderhouden is, waardoor het toegankelijk is voor alle teamleden, ongeacht hun technische achtergrond. Documentatie kan automatisch gegenereerd en geüpdatet worden door integratie met CI/CD pipelines, waardoor de documentatie altijd up-to-date blijft met de laatste wijzigingen in de codebase. TechDocs biedt krachtige zoekfunctionaliteit, zodat gebruikers snel de informatie kunnen vinden die ze nodig hebben. Omdat TechDocs is geïntegreerd met de Software Catalogus, kunnen gebruikers gemakkelijk toegang krijgen tot de documentatie van elke softwarecomponent direct vanuit de catalogus. Door de documentatie in de repository van de component te houden, profiteert het van versiebeheer, waardoor wijzigingen gemakkelijk kunnen worden bijgehouden en hersteld indien nodig.

\subsection{Voordelen}

Het gebruik van TechDocs binnen Backstage.io zorgt ervoor dat documentatie altijd up-to-date blijft dankzij de automatische generatie en integratie met CI/CD pipelines. Dit is essentieel voor betrouwbare technische informatie. De gebruiksvriendelijke Markdown-opmaak maakt het eenvoudig voor alle teamleden om bij te dragen aan en te profiteren van de documentatie. Het centraliseren van documentatie binnen dezelfde interface als de softwarecatalogus verhoogt de efficiëntie en gebruiksvriendelijkheid voor ontwikkelaars.

\section{Open Source Plugins}

Een van de krachtigste kenmerken van Backstage.io is de mogelijkheid om het platform uit te breiden met open-source plugins. Deze plugins voegen extra functionaliteiten toe en kunnen aangepast worden aan de specifieke behoeften van je organisatie.

\subsection{Kenmerken en Functionaliteiten}

Er is een breed aanbod aan beschikbare plugins, variërend van integraties met CI/CD tools tot monitoring en beveiligingstools. Deze plugins zijn meestal ontwikkeld door de gemeenschap en zijn open-source, wat betekent dat ze door iedereen kunnen worden gebruikt en aangepast. Organisaties kunnen eigen plugins ontwikkelen of bestaande plugins aanpassen om te voldoen aan hun specifieke behoeften, wat Backstage extreem flexibel en aanpasbaar maakt. Plugins kunnen eenvoudig worden geïntegreerd in het Backstage platform dankzij de modulaire architectuur, wat betekent dat nieuwe functionaliteiten snel en efficiënt kunnen worden toegevoegd zonder de kern van het platform aan te passen. Omdat veel plugins door de gemeenschap worden ontwikkeld, is er vaak uitgebreide ondersteuning en documentatie beschikbaar. De actieve gemeenschap rond Backstage zorgt ook voor continue verbetering en updates van de plugins.

\subsection{Voordelen}

Het gebruik van open-source plugins binnen Backstage.io maakt het platform zeer uitbreidbaar, waardoor nieuwe functionaliteiten eenvoudig kunnen worden toegevoegd en het platform toekomstbestendig is. De mogelijkheid om eigen plugins te ontwikkelen of bestaande plugins aan te passen biedt een hoge mate van flexibiliteit, zodat Backstage kan worden afgestemd op de specifieke behoeften van je organisatie. De actieve gemeenschap rond Backstage zorgt voor een continue stroom van nieuwe en verbeterde plugins, wat bijdraagt aan de waarde en relevantie van het platform.

\section{Conclusie}

Backstage.io biedt een krachtig platform voor het beheren van softwarecomponenten binnen een organisatie. Door middel van een centrale softwarecatalogus, gestandaardiseerde templates, geïntegreerde TechDocs en een uitgebreide set van open-source plugins, kunnen organisaties hun ontwikkelprocessen stroomlijnen en de productiviteit van hun teams verhogen. De flexibiliteit en uitbreidbaarheid van Backstage maken het een waardevolle tool voor elke moderne softwareontwikkelingsorganisatie.
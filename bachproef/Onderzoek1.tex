\chapter{Wolters Kluwer}
\label{ch:Onderzoek1}

In dit hoofdstuk word onderzocht welke bestaande plugins bruikbaar zijn en voor welke applicaties plugins moeten gemaakt worden.
\section{Core Features van Backstage.io}

Backstage.io biedt een gecentraliseerde interface voor ontwikkelteams om hun tools, gegevens en documenten te beheren. Hieronder bespreken we enkele essentiële plugins die ontwikkelaars kunnen helpen bij het gebruik van Backstage.io.

\subsection{TechDocs en Bitbucket Plugin}

\subsubsection{TechDocs}
TechDocs is de documentatieplugin van Backstage.io die het mogelijk maakt om documentatie rechtstreeks in Backstage te integreren. Dit is handig voor het bijhouden van technische documentatie, zoals API-beschrijvingen, handleidingen en andere ontwikkelingsgerelateerde documenten.

\subsubsection{Bitbucket Plugin}
Bitbucket is een Git-opslagplaatsbeheerder die door veel bedrijven wordt gebruikt voor broncodebeheer en samenwerking. De Bitbucket plugin voor Backstage.io integreert Bitbucket met het Backstage-platform, waardoor gebruikers gemakkelijk toegang hebben tot hun repositories, pull requests en andere functies binnen de Backstage-interface.

\subsection{Kubernetes Plugin}

De Kubernetes plugin voor Backstage.io stelt ontwikkelaars in staat om Kubernetes-clusters te beheren en te monitoren rechtstreeks vanuit Backstage. Dit omvat het bekijken van de status van pods, services, deployments en andere Kubernetes-resources. Deze integratie is essentieel voor teams die containers en microservices gebruiken, omdat het een gecentraliseerde manier biedt om hun infrastructuur te beheren.

\subsection{Backstage Search}

Backstage Search is een krachtige zoekfunctie die de efficiëntie van ontwikkelaars verhoogt door hen in staat te stellen snel toegang te krijgen tot de informatie die ze nodig hebben. Met deze functie kunnen gebruikers zoeken naar documentatie, API's, componenten, en andere resources binnen het Backstage-platform. Door verschillende bronnen te indexeren, zoals Git repositories, TechDocs, en meer, zorgt Backstage Search ervoor dat alle relevante informatie binnen handbereik is.

\subsection{Implementatie}
% proof of concept van section

\section{Jira en Jira Dashboard Plugin}

\subsection{Jira Plugin}

De Jira plugin voor Backstage.io integreert met Atlassian Jira, waardoor ontwikkelaars toegang hebben tot hun projecten, issues en sprints binnen Backstage. Dit bevordert een naadloze workflow door het centraliseren van projectmanagement- en ontwikkelingsactiviteiten.

\subsection{Jira Dashboard Plugin}

Met de Jira Dashboard plugin kunnen gebruikers aangepaste dashboards maken en bekijken die informatie en statistieken uit hun Jira-projecten weergeven. Dit helpt bij het monitoren van de voortgang en prestaties van teams en projecten.

\subsection{Implementatie}
% proof of concept van section

\section{JFrog Artifactory Plugins}

\subsection{Artifacts Plugin voor JFrog Artifactory}

De Artifacts plugin voor JFrog Artifactory integreert de artifact repository met Backstage.io, waardoor ontwikkelaars gemakkelijk toegang hebben tot hun build-artefacten, afhankelijkheden en andere opgeslagen resources.

\subsection{JFrog Artifactory Libs Plugin}

De JFrog Artifactory Libs plugin biedt specifieke functies voor het beheren van bibliotheken en afhankelijkheden binnen Artifactory. Dit is nuttig voor teams die gebruik maken van meerdere bibliotheken en ervoor willen zorgen dat alle afhankelijkheden goed beheerd en bijgehouden worden.

\subsection{Container Image Registry voor JFrog Artifactory}

Deze plugin maakt het mogelijk om container images te beheren binnen Artifactory vanuit Backstage. Ontwikkelaars kunnen images pushen, pullen, en beheren zonder Backstage te verlaten, wat de integratie en beheer van containergebaseerde toepassingen vereenvoudigt.

\subsection{Implementatie}
% proof of concept van section


\section{BlackDuck Plugin}

BlackDuck is een tool voor open source beveiliging en licentiebeheer. De integratie met Backstage.io stelt ontwikkelaars in staat om direct in Backstage inzicht te krijgen in de beveiliging en licenties van hun open source afhankelijkheden. Dit helpt bij het vroegtijdig identificeren van kwetsbaarheden en nalevingsproblemen, waardoor de algehele veiligheid van de software wordt verhoogd.

\subsection{Implementatie}
% proof of concept van section

\section{SonarQube Plugin}

SonarQube is een tool voor continue inspectie van codekwaliteit. De SonarQube plugin integreert deze tool met Backstage.io, waardoor ontwikkelaars inzicht krijgen in codekwaliteitsmetingen, zoals technische schuld, bugs, kwetsbaarheden en code duplicatie. Dit helpt bij het handhaven van hoge codekwaliteitsnormen en het verbeteren van de softwarekwaliteit.

Door deze plugins te integreren met Backstage.io, kan Wolters Kluwer de ontwikkelervaring voor hun teams verbeteren, de efficiëntie verhogen en een meer gestroomlijnde workflow creëren. Deze tools en plugins bieden een uitgebreid ecosysteem dat de samenwerking, beveiliging en het beheer van ontwikkelingsprojecten bevordert.

\subsection{Implementatie}
% proof of concept van section

\section{Dev Containers en DevTools Plugins}

\subsection{Dev Containers}

De Dev Containers plugin ondersteunt het gebruik van ontwikkelingscontainers, zoals die van Docker. Dit stelt ontwikkelaars in staat om consistente en reproduceerbare ontwikkelingsomgevingen te creëren en te beheren, wat de onboarding van nieuwe teamleden en het beheer van afhankelijkheden vereenvoudigt.

\subsection{DevTools}

DevTools plugin verzamelt verschillende ontwikkelingshulpmiddelen en stelt ze beschikbaar binnen Backstage. Dit kan tools omvatten voor debugging, profiling, logging, en meer, waardoor ontwikkelaars toegang hebben tot een breed scala aan hulpmiddelen die hun productiviteit verhogen.

\subsection{Implementatie}
% proof of concept van section

\section{TeamCity}

% TC plugin uitleg
\subsection{Implementatie}
% proof of concept van section

\section{Azure Plugins}

% WK slaat over naar Azure omgeving dus werd onderzocht

\subsection{Azure Pipelines}

De Azure Pipelines plugin integreert CI/CD (Continuous Integration/Continuous Deployment) pijplijnen van Azure DevOps met Backstage.io. Hierdoor kunnen ontwikkelaars hun build- en release-pijplijnen beheren en monitoren binnen Backstage.

\subsection{Azure Resources}

Deze plugin biedt inzicht in en beheer van verschillende Azure-resources zoals VM's, databases en netwerkcomponenten. Dit helpt ontwikkelaars om hun cloud-infrastructuur effectief te beheren vanuit een centrale Backstage-interface.

\subsection{Azure Sites}

Azure Sites plugin integreert Azure App Service met Backstage, waardoor ontwikkelaars hun web-apps en API's die op Azure draaien kunnen beheren en monitoren.

\subsection{Azure Spring Apps}

Deze plugin maakt het beheer van Spring Boot applicaties die op Azure Spring Apps draaien mogelijk vanuit Backstage. Het biedt functies zoals het starten, stoppen en schalen van applicaties.

\subsection{Azure Storage Explorer}

De Azure Storage Explorer plugin integreert Azure Blob Storage en andere opslagservices met Backstage, waardoor gebruikers hun opslagaccounts en containers kunnen beheren.

\subsection{Azure DevOps TechDocs}

Voor bedrijven die overschakelen naar Azure DevOps, biedt de Azure DevOps TechDocs plugin integratie met Backstage TechDocs. Hierdoor kunnen ontwikkelteams hun documentatie vanuit Azure DevOps Repositories synchroniseren en weergeven in Backstage.

\subsection{Implementatie}
% proof of concept van section

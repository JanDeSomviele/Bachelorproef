%==============================================================================
% Sjabloon onderzoeksvoorstel bachproef
%==============================================================================
% Gebaseerd op document class `hogent-article'
% zie <https://github.com/HoGentTIN/latex-hogent-article>

% Voor een voorstel in het Engels: voeg de documentclass-optie [english] toe.
% Let op: kan enkel na toestemming van de bachelorproefcoördinator!
\documentclass{hogent-article}

% Invoegen bibliografiebestand
\addbibresource{voorstel.bib}

% Informatie over de opleiding, het vak en soort opdracht
\studyprogramme{Professionele bachelor toegepaste informatica}
\course{Bachelorproef}
\assignmenttype{Onderzoeksvoorstel}
% Voor een voorstel in het Engels, haal de volgende 3 regels uit commentaar
% \studyprogramme{Bachelor of applied information technology}
% \course{Bachelor thesis}
% \assignmenttype{Research proposal}

\academicyear{2023-2024} 


\title{'Pre-flight check' oplossing voor software deployments}

\author{Jan De Somviele}
\email{jan.desomviele@student.hogent.be}

% TODO: Medestudent
% Gaat het om een bachelorproef in samenwerking met een student in een andere
% opleiding? Geef dan de naam en emailadres hier
% \author{Yasmine Alaoui (naam opleiding)}
% \email{yasmine.alaoui@student.hogent.be}


\supervisor[Co-promotor]{J. Delamper (Wolters Kluwers, \href{mailto:jan.delamper@wolterskluwer.com}{jan.delamper@wolterskluwer.com})}

% Binnen welke specialisatierichting uit 3TI situeert dit onderzoek zich?
% Kies uit deze lijst:
%
% - Mobile \& Enterprise development
% - AI \& Data Engineering
% - Functional \& Business Analysis
% - System \& Network Administrator
% - Mainframe Expert
% - Als het onderzoek niet past binnen een van deze domeinen specifieer je deze
%   zelf
%
\specialisation{System \& Network Administrator}
\keywords{Automatisering, Software-implementatie, Betrouwbaarheid}

\begin{document}

\begin{abstract}
  Hier schrijf je de samenvatting van je voorstel, als een doorlopende tekst van één paragraaf. Let op: dit is geen inleiding, maar een samenvattende tekst van heel je voorstel met inleiding (voorstelling, kaderen thema), probleemstelling en centrale onderzoeksvraag, onderzoeksdoelstelling (wat zie je als het concrete resultaat van je bachelorproef?), voorgestelde methodologie, verwachte resultaten en meerwaarde van dit onderzoek (wat heeft de doelgroep aan het resultaat?).
\end{abstract}

\tableofcontents

% De hoofdtekst van het voorstel zit in een apart bestand, zodat het makkelijk
% kan opgenomen worden in de bijlagen van de bachelorproef zelf.
%---------- Inleiding ---------------------------------------------------------

\section{Introductie}%
\label{sec:introductie}
% TO DO: uitleg Backstage.io

De vraag naar dit onderzoek komt omdat Wolters Kluwer wilt overschakelen naar een cloud omgeving en hun software ontwikkelingsplatform eventueel wil aanpassen. Deze bachelorproef richt zich op de ontwikkeling van een intern ontwikkelingsplatform met behulp van Backstage.io. Het platform fungeert als een centrale hub voor diverse ontwikkelingsprocessen en -tools. De focus ligt op het optimaliseren van de ontwikkeling, het stroomlijnen van processen en het verbeteren van de algehele efficiëntie van interne softwareontwikkeling. 
De kernvraag van dit onderzoek is hoe Backstage.io kan worden geïmplementeerd en aangepast binnen Wolters Kluwer als intern ontwikkelingsplatform om de betrouwbaarheid te verbeteren en de soepele implementatie van softwaretoepassingen te faciliteren. Bovendien zal een proof of concept worden geleverd om de praktische toepassing en haalbaarheid van Backstage.io binnen de context van Wolters Kluwer weer te geven.




%Formuleer duidelijk de onderzoeksvraag! De begeleiders lezen nog steeds te veel voorstellen waarin we geen onderzoeksvraag terugvinden.


%---------- Stand van zaken ---------------------------------------------------

\section{State-of-the-art}%
\label{sec:state-of-the-art}
% TO DO:huidige toestand in WoKlu, in referenties documentatie en ev. docs in wolters kluwer van huidige toestand

\cite{CK_BackstageGithub}
\cite{Backstage_Docs}

In Wolters kluwer wordt er nu gebruikt gemaakt van TeamCity, Jira en Confluence voor het development process. Omdat het bedrijf wilt overschakelen naar een cloud omgeving in plaats van de on premises omgeving van teamcity zoeken ze alternatieven. Backstage.io is een van deze alternatieven. Backstage.io is een open source spotify programma dat gemakkelijke toegang biedt aan de resources van developers. 

%---------- Methodologie ------------------------------------------------------
\section{Methodologie}%
\label{sec:methodologie}
\subsection{Literatuurstudie}
In deze fase wordt een uitgebreide literatuurstudie uitgevoerd volgens de CRAAP-methode om bestaande implementatiepraktijken van Backstage.io als intern ontwikkelingsplatform te onderzoeken. Hierbij wordt specifiek gekeken naar succesvolle aanpassingen en integraties in vergelijkbare organisaties. Deze fase heeft als resultaat een literatuurstudierapport waarin relevante informatie over de implementatie van Backstage.io wordt behandeld.

\subsection{Oplossingsontwerp }
Voer gesprekken met ontwikkelaars en belanghebbenden om de behoeften en vereisten voor de implementatie van Backstage.io als intern ontwikkelingsplatform vast te stellen. Definieer duidelijk de functionaliteiten en aanpasbaarheidsmogelijkheden, waarbij ontwikkelaars de flexibiliteit hebben om het platform aan te passen aan specifieke implementatievereisten. Deze fase heeft als deliverable een gedetailleerd ontwerp van de implementatie.

\subsection{Implementatie:}
Implementeer Backstage.io met aandacht voor de ontworpen functionaliteiten en aanpasbaarheid. Integreer het platform in de bestaande ontwikkelingsprocessen van Wolters Kluwer. Deze fase heeft als deliverable een functioneel intern ontwikkelingsplatform met Backstage.io, klaar voor verdere evaluatie.

\subsection{Proof of Concept}
Pas het geïmplementeerde Backstage.io-platform toe in een real-world scenario om de functionaliteit, efficiëntie en aanpasbaarheid te valideren binnen de specifieke context van Wolters Kluwer. Verzamel feedback van ontwikkelaars en belanghebbenden over de bruikbaarheid en effectiviteit van het platform. Het proof of concept-rapport geeft concrete inzichten en verbeteringsmogelijkheden.

\subsection{Evaluatie}
Schrijf een gedetailleerd rapport met een beschrijving van het geïmplementeerde Backstage.io als intern ontwikkelingsplatform. In dit rapport wordt ook de feedback van de proof of concept geanalyseerd en worden aanbevelingen gedaan voor verdere optimalisatie en implementatie in de praktijk. Deze fase heeft als deliverable een eindrapport en een implementatievoorstel.

%---------- Verwachte resultaten ----------------------------------------------
\section{Verwacht resultaat, conclusie}%
\label{sec:verwachte_resultaten}

Deze bachelorproef streeft naar de ontwikkeling van een intern ontwikkelingsplatform met behulp van Backstage.io. Verwacht wordt dat dit platform aanzienlijke voordelen zal bieden, waaronder efficiëntere ontwikkelingsprocessen, aanpasbaarheid aan de unieke omgeving van de organisatie, verbeterde algehele efficiëntie van interne softwareontwikkeling, en een gestroomlijnde integratie van diverse ontwikkelingstools.

Het ontwikkelde platform zal concrete resultaten opleveren, waarbij de functionaliteit en voordelen van Backstage.io worden benadrukt in het eindrapport en het implementatievoorstel. Het is belangrijk op te merken dat de verwachtingen als leidraad dienen en dat het onderzoek openstaat voor nieuwe inzichten, met specifieke aandacht voor het begrijpen van ontwikkelingsprocessen binnen de context van het gebruik van Backstage.io.



\printbibliography[heading=bibintoc]

\end{document}
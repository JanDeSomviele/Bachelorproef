%---------- Inleiding ---------------------------------------------------------

\section{Introductie}%
\label{sec:introductie}

Deze bachelorproef richt zich op de ontwikkeling van een intern ontwikkelingsplatform met behulp van Backstage.io. Het platform fungeert als een centrale hub voor diverse ontwikkelingsprocessen en -tools. De focus ligt op het optimaliseren van de ontwikkeling, het stroomlijnen van processen en het verbeteren van de algehele efficiëntie van interne softwareontwikkeling. Als onderdeel van dit onderzoek zal er speciale aandacht worden besteed aan de implementatie van Backstage.io binnen het bedrijf Wolters Kluwer. De kernvraag van dit onderzoek is hoe Backstage.io kan worden geïmplementeerd en aangepast binnen Wolters Kluwer als intern ontwikkelingsplatform om de betrouwbaarheid te verbeteren en de soepele implementatie van softwaretoepassingen te faciliteren. Bovendien zal er een proof of concept worden geleverd om de praktische toepassing en haalbaarheid van Backstage.io binnen de specifieke context van Wolters Kluwer te onderzoeken.




%Formuleer duidelijk de onderzoeksvraag! De begeleiders lezen nog steeds te veel voorstellen waarin we geen onderzoeksvraag terugvinden.


%---------- Stand van zaken ---------------------------------------------------

\section{State-of-the-art}%
\label{sec:state-of-the-art}



%---------- Methodologie ------------------------------------------------------
\section{Methodologie}%
\label{sec:methodologie}
\subsection{Literatuurstudie:}
In deze fase wordt een uitgebreide literatuurstudie uitgevoerd volgens de CRAAP-methode om bestaande implementatiepraktijken van Backstage.io als intern ontwikkelingsplatform te onderzoeken. Hierbij wordt specifiek gekeken naar succesvolle aanpassingen en integraties in vergelijkbare organisaties. De duur van deze fase is 2 weken, en het resultaat is een literatuurstudierapport waarin relevante informatie over de implementatie van Backstage.io wordt behandeld.

\subsection{Oplossingsontwerp: }
Voer gesprekken met ontwikkelaars en belanghebbenden om de specifieke behoeften en vereisten voor de implementatie van Backstage.io als intern ontwikkelingsplatform vast te stellen. Definieer duidelijk de functionaliteiten en aanpasbaarheidsmogelijkheden, waarbij ontwikkelaars de flexibiliteit hebben om het platform aan te passen aan specifieke implementatievereisten. Deze fase duurt 2 weken en heeft als deliverable een gedetailleerd ontwerp van de implementatie.

\subsection{Implementatie:}
Implementeer Backstage.io met aandacht voor de ontworpen functionaliteiten en aanpasbaarheid. Integreer het platform in de bestaande ontwikkelingsprocessen van Wolters Kluwer. Deze fase duurt 6 weken en heeft als deliverable een functioneel intern ontwikkelingsplatform met Backstage.io, klaar voor verdere evaluatie.

\subsection{Proof of Concept:}
Pas het geïmplementeerde Backstage.io-platform toe in een real-world scenario om de functionaliteit, efficiëntie en aanpasbaarheid te valideren binnen de specifieke context van Wolters Kluwer. Verzamel feedback van ontwikkelaars en belanghebbenden over de bruikbaarheid en effectiviteit van het platform. Deze fase duurt 4 weken, waarbij het proof of concept-rapport concrete inzichten en verbeteringsmogelijkheden omvat.

\subsection{Evaluatie:}
Schrijf een gedetailleerd rapport met een beschrijving van het geïmplementeerde Backstage.io als intern ontwikkelingsplatform. In dit rapport wordt ook de feedback van de proof of concept geanalyseerd en worden aanbevelingen gedaan voor verdere optimalisatie en implementatie in de praktijk. Deze fase duurt 1 week en heeft als deliverable een eindrapport en een implementatievoorstel.

%---------- Verwachte resultaten ----------------------------------------------
\section{Verwacht resultaat, conclusie}%
\label{sec:verwachte_resultaten}

Deze bachelorproef streeft naar de ontwikkeling van een intern ontwikkelingsplatform met behulp van Backstage.io. Verwacht wordt dat dit platform aanzienlijke voordelen zal bieden, waaronder efficiëntere ontwikkelingsprocessen, aanpasbaarheid aan de unieke omgeving van de organisatie, verbeterde algehele efficiëntie van interne softwareontwikkeling, en een gestroomlijnde integratie van diverse ontwikkelingstools.

Het ontwikkelde platform zal concrete resultaten opleveren, waarbij de functionaliteit en voordelen van Backstage.io worden benadrukt in het eindrapport en het implementatievoorstel. Het is belangrijk op te merken dat de verwachtingen als leidraad dienen en dat het onderzoek openstaat voor nieuwe inzichten, met specifieke aandacht voor het begrijpen van ontwikkelingsprocessen binnen de context van het gebruik van Backstage.io.

